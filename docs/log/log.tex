\documentclass[a4paper]{article}

\usepackage{graphicx}
\usepackage[linesnumbered,ruled,vlined]{algorithm2e}
\usepackage{color,soul}
\usepackage[utf8]{inputenc}
\usepackage[T1]{fontenc}
\usepackage{textcomp}
\usepackage{amsmath, amssymb}
\usepackage{caption}
\usepackage{listings}

% figure support
\usepackage{tikz}
\usetikzlibrary{calc}
\usepackage{import}
\usepackage{xifthen}
\pdfminorversion=7
\usepackage{pdfpages}
\usepackage{transparent}
\usepackage[hidelinks]{hyperref}
\usepackage{multirow}

\pdfsuppresswarningpagegroup=1

\begin{document}
	\title{Project log - Robotica}
	\author{Augello Andrea \and Castiglione Francesco Paolo \and La Martina Marco}
	\maketitle
	\tableofcontents

	\section{Setup}\label{sec:Setup}
	\begin{tabular}{|l|r|}
		\hline
		\multirow{2}{4em}{OS} & Ubuntu 18.04 \\
							  & Ubuntu 20.04 \\ \hline
		\multirow{2}{6em}{ROS version} & melodic \\
									   & noetic \\ \hline
		Webots & R2020b revision 1\\ \hline
		\multirow{2}{11em}{Target hardware} & Raspberry Pi 4B \\
											& Raspberry Pi 3B+ \\ \hline
	\end{tabular}

	\section{Name}\label{sec:Name}
	Our team has chosen the name \textbf{Change}, which resembles \textbf{Chang'e 4} \cite{change4}, the spacecraft mission part of the second phase of the Chinese Lunar Exploration Program, which achieved humanity's first soft landing on the far side of the moon.

	\section{Libraries}\label{sec:Libraries}
	
	% Bibliography
	\bibliographystyle{unsrt}
	\begin{thebibliography}{9}
		\bibitem{change4} 
		\textit{https://www.theguardian.com/science/2019/jan/03/china-probe-change-4-land-far-side-moon-basin-crater}. \newline
		The Guardian. 3 January 2019. Archived from the original on 3 January 2019. Retrieved 3 January 2019.
	\end{thebibliography}

\end{document}
