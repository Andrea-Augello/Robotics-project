\documentclass[a4paper]{article}

\usepackage{graphicx}
\usepackage[linesnumbered,ruled,vlined]{algorithm2e}
\usepackage{color,soul}
\usepackage[utf8]{inputenc}
\usepackage[T1]{fontenc}
\usepackage{textcomp}
\usepackage{amsmath, amssymb}
\usepackage{caption}
\usepackage{listings}

% figure support
\usepackage{tikz}
\usetikzlibrary{calc}
\usepackage{import}
\usepackage{xifthen}
\pdfminorversion=7
\usepackage{pdfpages}
\usepackage{transparent}
\usepackage[hidelinks]{hyperref}
\usepackage{multirow}

\pdfsuppresswarningpagegroup=1

\begin{document}
	\title{Project log - Robotica}
	\author{Augello Andrea \and Castiglione Francesco Paolo \and La Martina Marco}
	\maketitle
	\tableofcontents

	\section{Setup}\label{sec:Setup}
	\begin{tabular}{|l|r|}
		\hline
		\multirow{2}{4em}{OS} & Ubuntu 18.04 \\
							  & Ubuntu 20.04 \\ \hline
		\multirow{2}{6em}{ROS version} & melodic \\
									   & noetic \\ \hline
		Webots & R2020b revision 1\\ \hline
		\multirow{2}{11em}{Target hardware} & Raspberry Pi 4B \\
											& Raspberry Pi 3B+ \\ \hline
	\end{tabular}

	\section{Name}\label{sec:Name}
	Our team has chosen the name \textbf{Change}, which resembles \textbf{Chang'e 4} \cite{change4}, the spacecraft mission part of the second phase of the Chinese Lunar Exploration Program, which achieved humanity's first soft landing on the far side of the moon.

	\section{Libraries and environment}\label{sec:Libraries}
	We have used the \textbf{webots\_ros} \cite{webotsRos} package in order to gain deeper understanding of how to interface ROS nodes with the standard ROS controller for Webots. We have also studied the ROS documentation \cite{rosTutorial} in order to install and configure the ROS enviroment and also to understand fundamental ROS concepts related to nodes and topics.
	Moreover, we set-up the ROS interface in Webots following the cyberbotics documentation \cite{rosTutorial}.
	
	\section{Task}
	Our robot will be deployed in a room (such as the one showed in our demo) and its aim is to identify humans and estimate their relative positions. If the distance between said humans is less that a specified value, the robot will go towards them and invite them to respect social distancing (with both visual and audio output).
	
	\section{Tiago Iron} 
	The robot selected for the given task is the \textbf{TIAGo Iron}. \newline\textbf{PAL Robotics TIAGo Iron}\cite{tiagoiron} is a two-wheeled human-like robot with a torso and a head but no articulated arm. The model is a modular mobile platform that allows human-robot interaction.
	In order to achieve our goal, it was deemed necessary to add the following devices in the extension slot:
		\begin{enumerate}
			\item compass;
			\item gyro;	
			\item accelerometer;
		\end{enumerate}
	
	\section{ROS}

	
	
	\newpage
	% Bibliography
	\bibliographystyle{unsrt}
	\begin{thebibliography}{9}
		\bibitem{tiagoiron} 
		\textit{https://cyberbotics.com/doc/guide/tiago-iron}. \newline
		Webots TIAGo Iron documentation.
		\bibitem{change4} 
		\textit{https://www.theguardian.com/science/2019/jan/03/china-probe-change-4-land-far-side-moon-basin-crater}. \newline
		The Guardian, 3 January 2019.
		\bibitem{webotsRos} 
		\textit{https://github.com/cyberbotics/webots\_ros}. \newline
		Github page for the \texttt{webots\_ros} package from \textit{cyberbotics}.
		\bibitem{rosTutorial} 
		\textit{https://wiki.ros.org/ROS/Tutorials}. \newline
		ROS documentation from ROS.org.
		\bibitem{webotsRosSetup} 
		\textit{https://www.cyberbotics.com/doc/guide/tutorial-8-using-ros}. \newline
		Cyberbotics documentation.
	\end{thebibliography}

\end{document}
