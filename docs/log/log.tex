\documentclass[a4paper]{article}

\usepackage{graphicx}
\usepackage[linesnumbered,ruled,vlined]{algorithm2e}
\usepackage{color,soul}
\usepackage[utf8]{inputenc}
\usepackage[T1]{fontenc}
\usepackage{textcomp}
\usepackage{amsmath, amssymb}
\usepackage{caption}
\usepackage{listings}

% figure support
\usepackage{tikz}
\usetikzlibrary{calc}
\usepackage{import}
\usepackage{xifthen}
\pdfminorversion=7
\usepackage{pdfpages}
\usepackage{transparent}
\usepackage[hidelinks]{hyperref}
\usepackage{multirow}

\pdfsuppresswarningpagegroup=1

\begin{document}
	\title{Project log - Robotica}
	\author{Augello Andrea \and Castiglione Francesco Paolo \and La Martina Marco}
	\maketitle
	\tableofcontents

	\section{Setup}\label{sec:Setup}
	\begin{tabular}{|l|r|}
		\hline
		\multirow{2}{4em}{OS} & Ubuntu 18.04 \\
							  & Ubuntu 20.04 \\ \hline
		\multirow{2}{6em}{ROS version} & melodic \\
									   & noetic \\ \hline
		Webots & R2020b revision 1\\ \hline
		\multirow{2}{11em}{Target hardware} & Raspberry Pi 4B \\
											& Raspberry Pi 3B+ \\ \hline
	\end{tabular}

	\section{Name}\label{sec:Name}
	Our team has chosen the name \textbf{Change}, which resembles \textbf{Chang'e 4} \cite{change4}, the spacecraft mission part of the second phase of the Chinese Lunar Exploration Program, which achieved humanity's first soft landing on the far side of the moon.

	\section{Environment}\label{sec:Libraries}
	We have explored the \textbf{webots\_ros} \cite{webotsRos} package in order to gain deeper understanding of how to interface ROS nodes with the standard ROS controller for Webots. We have also studied the ROS documentation \cite{rosTutorial} in order to install and configure the ROS enviroment and also to understand fundamental ROS concepts related to nodes and topics.
	Moreover, we set-up the ROS interface in Webots following the cyberbotics documentation \cite{rosTutorial}.
	
	\section{Dependencies}
	This is a list of the libraries used in our project and a brief explanation of their relevance:
	\begin{itemize}
		\item opencv 4.x, a library aimed at computer vision\cite{opencv};
		\item imutils, series of convenience functions to make basic image processing functions such as translation, rotation, resizing, skeletonization\cite{imutils};
		\item sklearn, a machine learning library, featuring various classification, regression and clustering algorithms including support vector machines, random forests, gradient boosting, k-means and DBSCAN\cite{scikit};
		\item numpy, a library that adds support for large, multi-dimensional arrays and matrices, along with a large collection of high-level mathematical functions to operate on these arrays\cite{numpy};
		\item matplotlib, a comprehensive library for creating static, animated, and interactive visualizations\cite{matplotlib};
	\end{itemize}

	\section{Task}
	Our robot will be deployed in a room (such as the one showed in our demo) and its aim is to identify humans and avoid gatherings. It must estimate people's relative positions and, if the distance between said humans is less that a specified value, the robot will go towards them and invite them to respect social distancing (with both visual and audio output).
	
	\section{Tiago Iron} 
	The robot selected for the given task is the \textbf{TIAGo Iron}. \newline\textbf{PAL Robotics TIAGo Iron}\cite{tiagoiron} is a two-wheeled human-like robot with a torso and a head but no articulated arm. The model is a modular mobile platform that allows human-robot interaction.\newline
	We added a \textbf{speaker} and a \textbf{display} with a corresponding support solid to the base TIAGo model available in Webots.
	We also had to ask the Webots developers for the precise size of the \textbf{wheels} since the model does not exactly match the specifications given in the data TIAGo datasheet\cite{Tiago IRON datasheet} and we discovered that they are 200mm.
	\newline We also decided to modify the IMU in order to best fit our goals and use an IMU with 6 degrees of freedom.
	The IMU consists of the following components:
		\begin{enumerate}
			\item gyro;	
			\item accelerometer;
		\end{enumerate}
	We decided to \textbf{not use the compass} because in a real scenario it would have been subject to various degrees of interference (significantly more so than a gyro), especially in an environment with many metal objects.
	
	\section{Positioning}
	In order to obtain the linear motion from the IMU a double integral is applied to the signal. The mathematical reasoning behind such approach is to remember that acceleration is the rate of change of the velocity of an object. At the same time, the velocity is the rate of change of the position of that same object. Since integration is the opposite of the derivative, if we know the object's acceleration we can get the position through double integration.
	
	\section{Projection Matrix}
	\cite{OpenGL}
	
	\section{Object recognition}
	We evalued performance between YOLO V3, TinyYOLO, HoG , HoG + SVG  , HoG + SVG + NMS. Yolo wins because it is 443\% more efficient. Width and not height. Yolo yields much tighter bounding boxes. 
	
	\section{Clustering}
	We decided to lower the dimensionality of our data. We used cilindric coordinates and the feature vector is 2 dimensional.
	We used the Density-Based Scan with a threshold. The entities not belonging to the cluster are discarded.
	
	\section{ROS}
	
	\section{Bugs found in the Webots ROS Controller}
	Logical values did not allow callbacks.
	
	\newpage
	% Bibliography
	\bibliographystyle{unsrt}
	\begin{thebibliography}{19}
		\bibitem{tiagoiron} 
		\textit{https://cyberbotics.com/doc/guide/tiago-iron}. \newline
		Webots TIAGo Iron documentation.
		\bibitem{change4} 
		\textit{https://www.theguardian.com/science/2019/jan/03/china-probe-change-4-land-far-side-moon-basin-crater}. \newline
		The Guardian, 3 January 2019.
		\bibitem{webotsRos} 
		\textit{https://github.com/cyberbotics/webots\_ros}. \newline
		Github page for the \texttt{webots\_ros} package from \textit{cyberbotics}.
		\bibitem{rosTutorial} 
		\textit{https://wiki.ros.org/ROS/Tutorials}. \newline
		ROS documentation from ROS.org.
		\bibitem{webotsRosSetup} 
		\textit{https://www.cyberbotics.com/doc/guide/tutorial-8-using-ros}. \newline
		Cyberbotics documentation.
		\bibitem{Tiago IRON datasheet} 
		\textit{https://pal-robotics.com/wp-content/uploads/2019/07/Datasheet\_TIAGo\_Complete.pdf}. \newline
		Tiago IRON datasheet.
		\bibitem{OpenGL} 
		\textit{https://www.songho.ca/opengl/gl\_projectionmatrix.html}. \newline
		OpenGL Projection Matrix.
		\bibitem{positioning} 
		\textit{https://www.nxp.com/docs/en/application-note/AN3397.pdf}. \newline
		Implementing Positioning Algorithms Using Accelerometers.
		\bibitem{gmapping} 
		\textit{https://people.eecs.berkeley.edu/~pabbeel/cs287-fa11/slides/gmapping.pdf}. \newline
		Gmapping from UC Berkeley EECS, Pieter Abbeel.
		\bibitem{opencv} 
		\textit{https://opencv.org/}. \newline
		OpenCV Website.
		\bibitem{imutils} 
		\textit{https://github.com/jrosebr1/imutils}. \newline
		Imutils GitHub page.
		\bibitem{scikit} 
		\textit{https://scikit-learn.org/stable/}. \newline
		Scikit-learn website.
		\bibitem{numpy} 
		\textit{https://numpy.org/}. \newline
		Numpy website.
		\bibitem{matplotlib} 
		\textit{https://matplotlib.org/}. \newline
		Matplotlib website.
	\end{thebibliography}

\end{document}
